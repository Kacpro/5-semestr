\documentclass[11pt, a4paper]{article}

\usepackage[T1]{fontenc}
\usepackage[utf8]{inputenc}
\usepackage[polish]{babel}
\usepackage{listings}
\usepackage{mathtools}
\usepackage{blindtext}
\usepackage{scrextend}
\usepackage{graphicx}
\usepackage{hyperref}


\graphicspath{ {./images/} }


\begin{document}

\title{Teoria współbieżności\\laboratorium 1}
\author{Kacper Janda}
\maketitle

\section{Zadanie}
W systemie dziala \begin{math} N \end{math} wątkow, które dzielą obiekt licznika (początkowy stan licznika = \begin{math} 0 \end{math}). Każdy wątek wykonuje w pętli \begin{math} 5 \end{math} razy inkrementację licznika. Zakładamy, że inkrementacja składa się z sekwencji trzech instrukcji: read, inc, write (odczyt z pamięci, zwiększenie o \begin{math} 1 \end{math}, zapis do pamięci). Wątki nie są synchronizowane.

\begin{enumerate}

\item Jaka jest teoretycznie najmniejsza wartość licznika po zakończeniu działania wszystkich wątków i jaka kolejność instrukcji (przeplot) do niej prowadzi?\\\\
Najmniejsza teoretyczna wartość licznika wynosi \begin{math} 5 \end{math}. Powstaje ona na przykład gdy \begin{math} N-1 \end{math} wątków wykona wszystkie swoje operacje pomiędzy operacjami inc oraz write wątku \begin{math} N \end{math}-tego.\\

\item Analogiczne pytanie -- jaka jest maksymalna wartość licznika i odpowiedni przeplot instrukcji?\\\\
Największa możliwa wartość licznika wynosi \begin{math} 5N \end{math} i powstaje gdy wątki działają sekwencyjnie.

\end{enumerate}

\end{document}